\documentclass[12pt,a4paper]{article}

\usepackage{graphicx}
\usepackage{amsmath, amssymb}
\usepackage{float}
\usepackage{caption}
\usepackage{geometry}
\geometry{margin=1in}

\title{\textbf{Lightweight Cryptography - Toy Cipher Exploration}}
\author{Kriti Arora}
\date{November 2024}

\begin{document}

\maketitle

\section{Overview of the Toy Cipher}

The original Saturnin cipher operates on a 16×16 state, i.e., a matrix of 16 rows and 16 columns of 4-bit
cells, totaling 256 bits. In contrast, our toy variant significantly reduces the state size for analysis and
experimentation.

\textbf{Toy Cipher State.} The toy design uses an 8×4 state, consisting of:

\[
8\ \text{registers} \times 4\ \text{bits each} = 32\ \text{bits}.
\]

These registers are denoted
\[
(x_0, x_1, x_2, x_3, x_4, x_5, x_6, x_7),
\]
where each $x_i$ is a 4-bit value.

\begin{figure}[H]
    \centering
    \includegraphics[width=0.6\textwidth]{fig_saturnin_state.jpg}
    \caption{Full Saturnin state (16×16).}
\end{figure}

\begin{figure}[H]
    \centering
    \includegraphics[width=0.45\textwidth]{fig_toy_cipher_state.jpg}
    \caption{Toy cipher state (8 registers of 4 bits each).}
\end{figure}

\section{Nonlinear Layer: Bitsliced S-box Application}

The cipher state consists of eight 4-bit registers:
\[
S = (x_0, x_1, x_2, x_3, x_4, x_5, x_6, x_7),\quad x_i \in \{0,\ldots,15\}.
\]

\begin{figure}[H]
    \centering
    \includegraphics[width=0.55\textwidth]{fig_sbox_structure.jpg}
    \caption{Bitsliced S-box structure.}
\end{figure}

\subsection{Bitslice Structure}

The eight registers are divided into two groups:
\[
G_0 = (x_0, x_1, x_2, x_3), \quad
G_1 = (x_4, x_5, x_6, x_7).
\]

Each group contains four 4-bit words. For bit position $j \in \{0,1,2,3\}$, we define the slice
\[
(a_j, b_j, c_j, d_j) = ((u_0)_j, (u_1)_j, (u_2)_j, (u_3)_j).
\]

Thus, the design applies \textbf{8 S-boxes in parallel} — four from $G_0$ and four from $G_1$.

\subsection{Group-specific S-box Variants}

\begin{itemize}
    \item $S_0$ is applied to all slices of $G_0$.
    \item $S_1$ is applied to all slices of $G_1$.
\end{itemize}

\section{Shift Operations: SR\_slice and SR\_sheet}

\subsection{SR\_slice}

\begin{figure}[H]
    \centering
    \includegraphics[width=0.55\textwidth]{fig_sr_slice_before.jpg}
    \caption{SR\_slice: State before applying the slice shift.}
\end{figure}

\begin{figure}[H]
    \centering
    \includegraphics[width=0.55\textwidth]{fig_sr_slice_after.jpg}
    \caption{SR\_slice: Top row shifted left; bottom row unchanged.}
\end{figure}

\subsection{SR\_sheet}

\begin{figure}[H]
    \centering
    \includegraphics[width=0.55\textwidth]{fig_sr_sheet_before.jpg}
    \caption{SR\_sheet: Sheet before shifting.}
\end{figure}

\begin{figure}[H]
    \centering
    \includegraphics[width=0.55\textwidth]{fig_sr_sheet_after.jpg}
    \caption{SR\_sheet: Top row shifted; bottom row unchanged.}
\end{figure}

\section{MDS Layer}

\subsection{Original Saturnin MDS}

Each group $(t_0, t_1, t_2, t_3)$ undergoes:

\[
(t_0, t_1, t_2, t_3) \leftarrow (t_1, t_2, t_3, t_0 \oplus t_1).
\]

\subsection{Toy Cipher MDS}

Because the reduced state has 8 registers, each group contains two registers:

\[
(t_0, t_1) \leftarrow (t_1, t_0 \oplus t_1).
\]

\begin{figure}[H]
    \centering
    \includegraphics[width=0.6\textwidth]{fig_toy_mds.jpg}
    \caption{Toy cipher MDS.}
\end{figure}

\section{Empirical Properties of the MDS Layer}

\subsection{Correctness of the MDS and Its Inverse}

For input
\[
(0,0,0,1,0,0,0,0),
\]
MDS produced:
\[
(0,1,1,0,0,1,0,0),
\]
and inverse MDS recovered the original input.

\subsection{Diffusion Measurement}

Measured average diffusion:

\[
\textbf{3.38 output nibbles changed per input bit flip}.
\]

This indicates moderate diffusion, spreading each bit flip across multiple registers.

\end{document}
